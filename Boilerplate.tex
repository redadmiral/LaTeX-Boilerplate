%integrierte .bib Datei
\begin{filecontents}{bibliography.bib}
%Onlinegenerator für Zitationen: http://manas.tungare.name/software/isbn-to-bibtex/

@article{bergmannNovy,
author = {Bergmann, Knut; Novy Leonard},
title = {Zur Konkretisierung der Debatte über philanthropische Finanzierungsmodelle},
journal = {Journalismus in der digitalen Moderne},
pages = {201-2011},
year = {2013}
}

@book{bonfadelli2007,
   title =     {Medien und Migration: Europa als multikultureller Raum?},
   author =    {Heinz Bonfadelli and Heinz Moser (eds.)},
   publisher = {VS Verlag für Sozialwissenschaften},
   year =      {2007}
}

@online{latexGeschichte,
author = {Wielenga, Friso},
title  = {Latex - Geschichte},
date   = {2017-02-05},
url    = {http://www.selflinux.org/selflinux/html/latex_geschichte01.html}
}

\end{filecontents}

%Präambel

\documentclass[a4paper,11pt]{article}% Alternativ book oder report
\usepackage[german, ngerman]{babel} % Neue deutsche Rechtschreibung, deutsche Überschriften und Beschreibungen
\usepackage[T1] {fontenc} % Symbole
\usepackage[utf8]{inputenc} % UTF-8 Sonderzeichen
\usepackage[babel,german=quotes]{csquotes} % Deutsche Anführungszeichen
\usepackage{amsfonts}
\usepackage{listing}
\usepackage{graphicx}%ermöglicht es Grafiken einzubinden
\usepackage{eurosym}%Fügt das €-Symbol hinzu (\euro)

\usepackage[
   backend=biber      % Notwendig!
  ,style=verbose-ibid   % Stil des Literaturverzeichnisses (https://de.sharelatex.com/learn/Biblatex_bibliography_styles)
%  ,citestyle=ieee     % Stil der Zitationen
  ,babel=other        % Mehrere Sprachen im Literaturverzeichnis
%  ,sortlocale=de_DE   % locale of main language, it is for sorting
%  ,bibencoding=UTF8   % Gibt das Encoding der externen .bib Datei an. % löschen wenn externe .bib-Datei verwendet wird.
  ,block=space
]{biblatex}

\bibliography{bibliography} %interne .bib Datei
%\addbibresource{test2.bib} %externe .bib Datei

%Dokument

\title{\LaTeX -Boilerplate}
\author{Marco Lehner \textbackslash  lehnerma61237@th-nuernberg.de}
\date{\today}

\begin{document}
\maketitle
\newpage
\tableofcontents
\newpage

%Bibliographie
\newpage
\nocite{} %Stellt * ins Literaturverzeichnis, obwohl nicht zitiert
\printbibliography
\end{document}


%Anmerkungen
%
%Zitate in Fußnote und Literaturverzeichnis mit \footfullcite{}
%
%PDF in Konsole erstellen:
%
%latex dokument.tex
%biber dokument.bcf
%pdflatex dokument.tex
%
%eingeben. Dann die erstellte PDF-Datei öffnen.
%
%
% Ein Leitfaden zur Erstellung von Bibliographien:
%http://biblatex.dominik-wassenhoven.de/download/DTK-2_2008-biblatex-Teil1.pdf
%http://biblatex.dominik-wassenhoven.de/download/DTK-4_2008-biblatex-Teil2.pdf
% Teil 1 führt in Biblatex ein, Teil 2 zeigt praktische Anwendungen
%
%Troubleshooting:
%Bei Fehler "biblatex.sty nicht gefunden":
%"which latex" in die Konsole eingeben. Wenn eine andere Ausgabe als "/usr/local/texlive/2014/bin/x86_64-linux/latex" kommt, den Pfad mit "export PATH=/usr/local/texlive/2014/bin/x86_64-linux:$PATH" eingeben. Sollte gehen dann.
%
%Latexhandbuch (deutsch) http://mirror.selfnet.de/tex-archive/info/translations/biblatex/de/biblatex-de.pdf
%Latexeinführung der Fernuni Hagen: https://www.fernuni-hagen.de/imperia/md/content/zmi_2010/a026_latex_einf.pdf
